\documentclass[10pt,a4paper]{article}
\usepackage[utf8]{inputenc}
\usepackage[english]{babel}
\usepackage{amsmath}
\usepackage{amsfonts}
\usepackage{amssymb}
\usepackage{graphicx}
\usepackage{natbib}
\usepackage[left=3.8cm,right=3.8cm,top=1.5cm,bottom=2cm]{geometry}
\title{A Continuous-Time Dynamical System Describing both Rate Encoding and Spiking Neurons}

\begin{document}
\renewcommand{\refname}{}
\pagenumbering{gobble}
\maketitle

\noindent
Schubert, F. (1) \& Gros, C. (1)
\\
\bigskip

\noindent
1 Goethe University Frankfurt, Institute for Theoretical Physics, Frankfurt am Main, Germany
\\
Corresponding author: fschubert@th.physik.uni-frankfurt.de
\bigskip

Ever since the electro-physiological properties of neurons have 
been investigated, an abundance of dynamical systems was proposed 
and used to explain neuronal activity. The simplest types of models 
that exhibit spiking behavior by means of completely differentiable 
dynamics are nonlinear two-dimensional oscillators with two very distinct
time scales, whose dynamics are often derived by simplifying higher-dimensional 
systems by means of projection or separation of timescales \cite{Fitzhugh_1961,Morris_1981}. 
Furthermore, on an higher level of abstraction, we find neuronal models 
that encode information on the level of neuronal firing rates \cite{Gerstner_2002}.

We introduce a two-dimensional nonlinear system, modeling a wide range of 
dynamic properties of spiking neurons. Yet, by altering key parameters of this system, 
its dynamics become identical to those of a time-continuous rate encoding model. Thus, 
our model allows for a continuous transition between spiking dynamics 
and a canonical form of rate encoding. Thereby, the differences  of the dynamical 
properties of single units as well as of network structures under these two 
regimes can be treated within the same mathematical framework. In particular, 
applying this approach to recurrent networks enables us to unify concepts of 
irregular/regular network behavior typically associated with either of 
these classes of neuron models \cite{Sompolinsky_1988,Vreeswijk_1996,Brunel_2000}.

Furthermore, we show how a canonical form of Hebbian plasticity can be incorporated 
such that well-established formulations appear within the respective parameter
range, spanning from spike-timing dependent plasticity to a rate-based mechanism.

Though appearing contradictory, we believe that a \emph{unification} of different neural 
modeling schemes can help clarifying \emph{differences} between their functional properties,
since an analysis of the systems can take place within the same scope of qualitative 
and quantitative measures.
\bibliographystyle{unsrt}
\bibliography{bibfile_abstract}
\end{document}